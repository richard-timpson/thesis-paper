\chapter{Experiments}\label{ch:ch4}

\section{Model and Experiments}
\begin{itemize}
    \item Due to the open-source nature of virtuosoNet project and its attempt to build a more cohesive EPG model by introducing the pedal as an expressive feature and training on a much larger dataset, we built off of this model. 
    \item Because of the significant advances in other sequence modeling domains (such as NLP) and the indication of increased performance of another related task with the Music Transformer \cite{huang2018music}, the main question we want to answer is whether we can see similar increases in model performance by applying a Transformer ANN architecture to the problem domain. 
    \item We will experiment with a transformer encoder only architecture similar to BERT. The problem includes a 1-1 to mapping between every note in the score and a related note in a performance. This is different than seq-2-seq modeling problem such as neural machine translation which maps a sequence of one length to another sequence of a different length, which is what the full Transformer architecture was intended for. The Transformer Encoder can be seen as as a large encoder that learns the best representation for a given feature set. The model we'll build will use a simple FFNN that accepts the output of the transformer encoder to decode this representation and give the final feature set which is then used to create a performance. This is similar to the BERT architecture and it's intended application. \rtodo[,inline]{Come up with a more detailed explanation of this modeling choice. Also create a visual diagram that explains the transformer encoder with the simple regression model sitting on top of it}
    \item Because we are using the same dataset used to train virtuosoNet, we will directly compare the performance a Transformer model to the existing virtuosoNet models using the same quantitative metric, MSE. \rtodo[,inline]{Come up with specific model experiments and comparison in a table. Table doesn't have to have results but needs the general outline that will be used in the final paper}
    
\end{itemize}

\section{Evaluation}
\begin{itemize}
    \item Quantitative: Because we are using the same dataset used to train virtuosoNet, we will directly compare the performance a Transformer model to the existing virtuosoNet models using the same quantitative metric, MSE. \rtodo[,inline]{Come up with specific model experiments and comparison in a table. The table doesn't have to have results but needs the general outline that will be used in the final paper}
    \item Due to time and resource constraints, no sophisticated qualitative evaluation was conducted for the models. However, a personal evaluation was used during the entire model development process. \rtodo[,inline]{Talk about method used for personal analysis}
    \item 
\end{itemize}
