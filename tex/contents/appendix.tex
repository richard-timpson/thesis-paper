\appendix

	% Add your appendices here. You must leave the appendices enclosed in the appendices environment in order for the table of contents to be correct.
	
\begin{appendices}

\chapter{Experiment Results}

% Table \ref{tab:quantitative} shows all of the results for all of the models that we trained, as well as results from a selection of \vnet{} models. The left side of the table presents the configuration for each of the models. The \vnet{} models (shown at the bottom) presented in other works~\cite{jeong2019virtuosonet, jeong2019graph} don't have any comparable hyperparameter values, and so only the experiment results are given for these models.   Because we used Neptune AI to manage our experiments, we identify each of our models using the experiment's Neptune ID, denoted as \nep{}. \nl{} is the number of layers, \dhid{} is the dimension of the hidden layers, \drop{} is the dropout, \lr{} is the learning rate, \clip{} is the gradient clip, and \nh{} is the number of attention heads. 

% The table's right side presents the MSE results for all models along the five different expressive dimensions mentioned in \ref{sec:qualitative-eval-problems} and the total MSE, which is an aggregation of all the individual expressive features. The entries for the HAN models come from virtuosoNet and are given in \cite{jeong2019virtuosonet}. The values for the best performing models of each family are bolded. T-BL, T-Best, and T-Worst indicate the baseline, best, and worst performing Transformer models, respectively. 


\chapter{Appendices I} \label{app:appendix_one}
\section{Musical Concepts and Terminology} \label{ase:app_one_sect_1}

\subsection{Pitch}

The first and most basic component in music is pitch. Pitch is a perceptual property of sounds that relates to the physical frequency of a sound vibration \cite{klapuri2006introduction}. It is what determines whether or not a sound can be thought of as ``high'' or ``low''. The most commonly known way to conceptualize pitch is the 88 different keys on a piano keyboard, where each key represents a different patch value. Pitch is most commonly labeled using scientific pitch notation, which couples a range of letters (A to G) with a range of numbers (zero to eight) that correspond to different octave ranges \footnote{\url{https://en.wikipedia.org/wiki/Scientific_pitch_notation}}. The most well known pitch is C4, or ``middle C'', and lays in the very center of a standard 88 key piano. \rtodo{Create or find visualization}

\subsection{Tempo and Timing}

Tempo in music describes the rate at which notes are played, and timing describes when a particular note should be played relative to the start of the composition. They are best explained in the context of modern western musical notation introduces the idea of note durations, time signatures, measures, and beats \footnote{See \url{https://en.wikipedia.org/wiki/Musical_notation\#Modern_staff_notation} for a more detailed explanation}. \rtodo[,inline]{Find a more intuitive way to explain this. The piano roll explanation and visualization may work better} Each composition is broken down into a sequence of measures, and the time signature defines how many beat exist per measure, as well as the duration of a single beat. For example, a 4/4 time signature indicates that there are 4 beats per measure (the top half of the time signature), and that the duration of each beat is represented by a quarter note. A 3/4 time signature would indicate only 3 beats per measure, with the beat duration represented by a quarter note. The timing of a note would refer to it's measure, beat, and note duration. Tempo is most commonly given in beats per minute (BPM). A composition with a 4/4 signatue and a 120 BPM would mean that after one minute, 30 measures of the composition should have been played so far. \rtodo{create or find visualization}

\subsection{Dynamics}

Dynamics can simply be thought of as how loud or soft a note should be played (or has been played). 

\section{Data Representation} \label{ase:app_one_sect_2}
% \section{Data}
% A brief section about the data used for the problem. Introduce MusicXML and MIDI
% \begin{itemize}
%     \item MusicXML
%     \begin{itemize}
%         \item A text based representation of a musical score. 
%         \item Created as a way to standardize score data among different notation software. 
%         \item Useful for EMP research because of the standardized format. 
%         \item Contains all relevant information about the score and it's related features. \rtodo{Add reference to feature section}
%     \end{itemize}
%     \item MIDI 
%     \begin{itemize}
%         \item Event based protocol for digital representation of musical instruments. 
%         \item Used in a variety of ways, most commonly known for it's use in DAW software to represent easily editable tracks for music production. 
%         \item Can be synthesized in many different ways. 
%         \item Contains all of the needed information to represent a musical performace. \rtodo{Reference feature section}. 
%     \end{itemize}
    
% \end{itemize}
\subsection{MusicXML}
\subsection{MIDI}
% \chapter{Second Appendix} \label{app:appendix_two}
\end{appendices}