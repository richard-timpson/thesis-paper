\appendix

	% Add your appendices here. You must leave the appendices enclosed in the appendices environment in order for the table of contents to be correct.
	
\begin{appendices}
\chapter{Appendices I} \label{app:appendix_one}
\section{Musical Concepts and Terminology} \label{ase:app_one_sect_1}

\subsection{Pitch}

The first and most basic component in music is pitch. Pitch is a perceptual property of sounds that relates to the physical frequency of a sound vibration \cite{klapuri2006introduction}. It is what determines whether or not a sound can be though of as "high" or low". The most commonly known way to conceptualize pitch is the 88 different keys on a piano keyboard, where each key represents a different patch value. Pitch is most commonly labeled using scientific pitch notation, which couples a range of letters (A to G) with a range of numbers (zero to eight) that correspond to different octave ranges \footnote{\url{https://en.wikipedia.org/wiki/Scientific_pitch_notation}}. The most well known pitch is C4, or "middle C", and lays in the very center of a standard 88 key piano. \rtodo{Create or find visualization}

\subsection{Tempo and Timing}

Tempo in music describes the rate at which notes are played, and timing describes when a particular note should be played relative to the start of the composition. They are best explained in the context of modern western musical notation introduces the idea of note durations, time signatures, measures, and beats \footnote{See \url{https://en.wikipedia.org/wiki/Musical_notation\#Modern_staff_notation} for a more detailed explanation}. \rtodo[,inline]{Find a more intuitive way to explain this. The piano roll explanation and visualization may work better} Each composition is broken down into a sequence of measures, and the time signature defines how many beat exist per measure, as well as the duration of a single beat. For example, a 4/4 time signature indicates that there are 4 beats per measure (the top half of the time signature), and that the duration of each beat is represented by a quarter note. A 3/4 time signature would indicate only 3 beats per measure, with the beat duration represented by a quarter note. The timing of a note would refer to it's measure, beat, and note duration. Tempo is most commonly given in beats per minute (BPM). A composition with a 4/4 signatue and a 120 BPM would mean that after one minute, 30 measures of the composition should have been played so far. \rtodo{create or find visualization}

\subsection{Dynamics}

Dynamics can simply be thought of as how loud or soft a note should be played (or has been played). 

\section{A2} \label{ase:app_one_sect_2}
% \chapter{Second Appendix} \label{app:appendix_two}
\end{appendices}