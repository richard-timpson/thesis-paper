\chapter{Conclusion} \label{ch:ch7}

\begin{itemize}
    \item Re-iterate high-level results. Transformer performs worse according to a quantitative metric. Doesn't necessarily mean that it's a "worse" model 
    \item Give a more in-depth discussion of the challenges of the problem domain with added context given the technical details discussed in our paper.
    \begin{itemize}
        \item Large need for more high-quality data. There are efforts on this front \cite{foscarin2020asap} - our model could easily apply to this dataset. I have the intuition that the alignment and consistency in the score data extremely important to building more robust and general models. The dataset used is large which adds it's own advantage, but it inherently presents more room for error. 
        \item Better methods for evaluation. This is a current pen question in EPG research, and could constitute an entire area of study outside of build the EPG models. Better evaluation methods would provide more intuition on how the Transformer model performs in comparison with other models. 
        \item Modeling of pedal appears to be a hard problem to solve, and hasn't been done outside of virtuosoNet and this work.
    \end{itemize}
    \item Future research directions
    \begin{itemize}
        \item Experiment with different transformer architectures. Immediate experiment to run is mess with the positional encodings as was done in \cite{huang2018music}
        \item Build and find better data 
        \item Explore evaluation methods
        \item See how Transformer model fits into other EPG frameworks and datasets, first example being the BM framework proposed by \cite{eduardo2018computational}. Also would be useful to apply it to other genres besides solo classical piano
    \end{itemize}
    \item Research Applications 
    \begin{itemize}
        \item Tutor systems 
        \item It's place in a full end to end music generation system. 
        \item Direct application to notation software and the creative process of composers. 
    \end{itemize}
    \item Philosophical questions about problem domain: How can a EPG can help to improve our understanding of music itself. If we can learn to generate human-like musical performances, does that mean we understand what it is that constitutes the "human" element of music? Another even more philosophical question would be - can we understand what it means to "human" in general. Music is one of, if not the most, unique human experiences. If building an EPG model can't directly answer the question of what it means to be human, it can at least provide insight. 
\end{itemize}
