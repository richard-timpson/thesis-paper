\chapter{Analysis} \label{ch:ch6}
Our initial experiments using a quantitative evaluation indicated that our Transformer model does not improve the state-of-the-art in EMP generation. As a method of "debugging" our Transformer model to look for potential errors and improvements, we wanted to listen to the performances generated by our models to identify any problems, if they existed at all. 

These listening tests acted both as an error analysis as well as our qualitative evaluation of the models. It became apparent to us right away that there was a disconnect between the quality of the model according to the quantitative metric and our evaluation. In this chapter, we present this analysis. The analysis includes our quantitative evaluation method, new experimental methods to improve our models, and conclusions about each model's quality according to our listening tests. 


%tm for transformer model

\section{Identifying Training Problems}\label{sec:qualitative-eval-problems}
Listening tests require generated performances, which we produced by using the \vnetf{} to create MIDI files given our models' feature output. We generated performances for several of our trained models using six compositions from three different composers, listed in table \ref{tab:compositions}. For easy comparison of each model's performance, we loaded their relevant MIDI files into the Digital Audio Workstation (DAW) Logic Pro X. We used the "Steinway Grand" instrument synthesizer to create the audio. Logic Pro X also provides visualizations of different performance aspects, including the MIDI velocity of each note and the sustain and soft pedals' value throughout the performance (see table \ref{fig:pedal-difference} for example pedal visualizations)%
\footnote{Performances generated by models can be view through our \href{https://ui.neptune.ai/richt3211/thesis/experiments}{Neptune Project}. Each experiment has an ID and we've run performance generation code for many of the models. To listen to performances, visit an experiments' artifacts tab and download available performance MIDI files. These MIDI files can be played in any DAW software, such as Logic Pro X. If performances don't exist for an experiment, please contact the author}. 

\begin{table}
    \setlength{\extrarowheight}{3pt}
    \begin{center}
    \begin{tabular}[]{| l | l |}
        \hline
        Composer & Composition \\ 
        \hline 
        Bach & Prelude in E Minor, BWV 855 \\
        Bach & Prelude in F-sharp Major, BWV 858  \\ 
        Chopin & Etude Op. 10, No. 12 \\ 
        Chopin & Fantaisie-Imropmptu \\ 
        Beethoven & Piano Sonata No. 17 First Movement \\ 
        Mozart & Piano Sonata No. 11 First Movement \\ 
        \hline
    \end{tabular}
    \caption{The compositions used for the qualitative evaluation of our models. All scores come in the form of MusicXML from MuseScore. None of the scores were present in the training data}
    \label{tab:compositions}
    \end{center}
\end{table}

As already mentioned, it was apparent that there was a mismatch between a model's performance quality according to the MSE value and our evaluation. For example, the Transformer model with $N_{id}$ 125 (which we will denote as \tm{125}, see Table \ref{tab:quantitative}) had much worse validation MSE metrics than almost every other model Transformer model. However, our listening test revealed that the absolute tempo for smaller models, such as the Transformer baseline \tm{147}, was much faster and sounded worse (to the point where the performances are almost ``unlistenable'') than \tm{125}. The potential disconnect between the quality of the model as determined by quantitative and qualitative evaluation led us to investigate possible problems with the training methods used by~\citet{jeong2019virtuosonet}. 

One of the first potential problems we identified was the MSE loss and evaluation function. The output features of virtuosoNet are a sequence of vectors with a length of 11. The first four features correspond to a single component of expression. They are tempo, velocity, deviation, and articulation, respectively. The last seven features are all different values that correspond to information about the sustain pedal~\cite{jeong2019score}.~\citet{jeong2019virtuosonet} present MSE metrics for five different expressive parameters, which include all of those previously mentioned, as well as the pedal. Both Jeong's and our references to pedal MSE are an aggregation of the seven different features that contain pedal information. In contrast, the other 4 MSE metrics only apply to a single feature. See table \ref{tab:quantitative} for the results in MSE for each different expressive parameter.

The original MSE used to train \vnet{} (which we call \vnet{} MSE) assumed that every individual feature of the output vector contributed equally to the final output and corresponding loss optimization. The pedal parameter has seven times as much information as every other parameter, and so the \vnetf{} MSE places more importance on pedal quality over every other feature. Given that the \vnetf{} MSE loss is used both for model training and evaluation, this means that it inherently rewards those models which express pedal better than all other performance features. If the \vnet{} models outperformed our Transformer models according to this metric, does that mean that they are better at modeling EMP generation as a whole, or that they are simply more fit to model the pedal? In an attempt to answer this question, we came up with a new weighted MSE loss function that allows for a more fair representation of the evaluation method. 

\newcommand{\rvec}[1]{\mb{#1}}

% vn for vector normal. predicted
% vh for vector hat. target
\newcommand{\vn}{\rvec{v}}
\newcommand{\vh}{\rvec{\hat{v}}}

% df for difference
\newcommand{\df}[1]{(\vn_{#1} - \vh_{#1})^2}

% al for alpha
\newcommand{\al}[1]{\alpha_{#1}}

We define the output vector as an 11 dimensional vector 
\begin{align*}
\rvec{v} = \{t, v, d, a, p_0, p_1, p_2, p_3, p_4, p_5, p_6\}   
\end{align*}
where $t$, $v$, $d$, and $a$ represent tempo, velocity, deviation, and articulation respectively, and $p_i$ represents a single component of the pedal. For a predicted output vector $\vn$ and the target output vector $\vh$, \vnetf{} MSE loss is 
\begin{align*}
MSE(\vn, \vh) = \frac{1}{n}\sum_{i=1}^{n}(\vn_i - \vh_i)^2   
\end{align*}
This can also be re-written as 
\begin{align*}
MSE(\vn, \vh) = \frac{1}{11}[\df{t} + \df{v} + \df{d} + \df{a} + \sum_{i=1}^{7}(\vn_{p_{i}} - \vh_{p_{i}})^2]   
\end{align*}
We introduce 5 different weight values: $\al{t}, \al{v}, \al{d}, \al{a}$ and $\al{p}$. Our weighted MSE loss is defined as 
\begin{align*}
W_{MSE}(\vn, \vh) = \frac{1}{\al{t} + \al{v} + \al{d} + \al{a} + \al{p}}[\al{t}\df{t} + \al{v}\df{v} + \al{d}\df{d} + \\
\al{a}\df{a} + \al{p}\sum_{i=1}^{7}(\vn_{p_{i}} - \vh_{p_{i}})^2 ]   
\end{align*}
\vnetf{} MSE can be seen as the weighted MSE with $\al{t}, \al{v}, \al{d}, \al{a} = 1$, and $\al{p} = 7$. 

This loss formulation shows the heavy bias towards the pedal feature. Is this is the right way to conceptualize what a `good' performance is? Would a different configuration of the expressive feature weights lead to a better outcome? Answering such questions in a straightforward and `objective' according to a quantitative metric is not plausible with our weighted MSE. Every different configuration of weights changes the evaluation metric itself, and therefore any direct metric comparison across experiments is meaningless. Knowing this, we continued to experiment with different weight configurations relying on our qualitative evaluation to determine performance quality. We present our observations from listening to model performances, with and without the change in MSE. 

% TODO: Move some of these sentiments to the discussion/conclusion sections. 
% Answering these questions is non-trivial. Therefore, we call to question the validity of using MSE as an evaluation metric, especially as presented in the \vnetf{}. With this in mind, we ran additional experiments changing the weights for each expressive parameter to see what the effects on performance would be according to our evaluation. 


\section{Qualitative Evaluation}\label{sec:qualitative-analysis}
Our first general observation given our new loss evaluation is that the two most important factors for overall performance are tempo and pedal. Performances with either of these two features outside of certain bounds make performances unlistenable. Suppose a performance's global tempo is too fast, and every other expressive parameter renders correctly. In that case, the resulting performance will still sound bad enough that it's not worth listening to at all \footnote{See \href{https://ui.neptune.ai/richt3211/thesis/e/THESIS-86/artifacts}{\tm{86} Fantaisie Impromptu} and \href{https://ui.neptune.ai/richt3211/thesis/e/THESIS-126/artifacts}{\tm{126} Etude Op. 10 No. 12}.}. We noticed a similar phenomenon with the pedal. Some models generated performances with the sustain pedal applied at all times with very few breaks. The result is a performance that sounds "muddied" and unrefined. Although these performances are more bearable than those with extreme tempo, they are still hard to listen to in any meaningful way \footnote{See \href{https://ui.neptune.ai/richt3211/thesis/e/THESIS-125/artifacts}{\tm{125} Piano Sonata 11} }. 

% lm for lstm model
\newcommand{\lm}[1]{$LSTM_{N_{#1}}$}

We also notice that the tempo and timing of the Transformer models are more dynamic than the LSTM models, both for our LSTM baseline and the \vnet{} models. For some models, the variability in timing seemed to be a good thing, while for others, it was so bad that it almost sounded like the model was still ``learning'' how to play. The tempo for all LSTM based models (except for some slight variations in the performances from HAN-M \footnote{Performances for the HAN-M are available at \href{https://ui.neptune.ai/richt3211/thesis/e/THESIS-162/artifacts}{$N_{126}$}}) was extraordinarily consistent and non-changing to the point of sounding robotic and mundane% 
\footnote{The resulting mundande sound is similar to a deadpan performance}. On one extreme with Transformer models, the highly dynamic tempo at times sounds like a real performer making mistakes \footnote{See \href{https://ui.neptune.ai/richt3211/thesis/e/THESIS-86/artifacts}{$T_{N_{86}}$ Piano Sonata 11}}. The other extreme is performances of some LSTM models that are so mundane they don't sound ``human'' at all \footnote{See \href{https://ui.neptune.ai/richt3211/thesis/e/THESIS-123/artifacts}{\lm{123} Piano Sonata 11}}. 

The pedaling in general of all models was mediocre at best. This observation is consistent with those of~\citet{jeong2019virtuosonet}. There are models with decent performance pedaling quality that somewhat follows the natural cadence of the composition but still do not match pedal in actual human performance. %Figure \ref{fig:pedal-difference} shows a visual comparison of the sustain pedal usage in different performances. 

% \begin{figure}
%     \centering
%     \missingfigure{Images that show the difference of 3 performances, all of the same composition. One performance should have really bad pedal, another should have mediocre pedal, and the other (a human performance) should have natural pedal. Will be gathered with screenshots from Logic Pro}
%     \caption{Test Caption}
%     \label{fig:pedal-difference}
% \end{figure}

% The importance of tempo and pedal was part of the intuition that led to our formulation of the weighted MSE by expressive parameter defined in \ref{sec:qualitative-eval-problems}. 
Our first experiments with the weighted MSE used an even weight distribution. We ran a few additional experiments with the tempo and pedal weights higher than others, along with some model size changes. Table \ref{tab:qualitative-models} shows a full description of the models and their parameters.

% \newcommand{\nep}{$N_{id}$}
% \newcommand{\mn}{$M$} % mn for 'model name'
% \newcommand{\nl}{$L$} % nl: num layers
% \newcommand{\dhid}{$d_{hid}$} % dhid: dimension hidden size
% \newcommand{\drop}{$D$} % D: Dropout
% \newcommand{\lr}{$LR$} % LR: Learning Rate
% \newcommand{\clip}{$C$} % C: gradient clip
% \newcommand{\nh}{$H$} % nh: num heads
\newcommand{\am}{$AM$}

\begin{table}
    \setlength{\extrarowheight}{3pt}
    \begin{center}
    \begin{tabular}[]{| c | c c c | c c c c c |}
        \hline
        \multicolumn{5}{|c|}{Model Configuration} & \multicolumn{4}{c|}{Expressive Weights}\\
        \hline
        \nep & \nl & \dhid & \nh & $\al{t}$ & $\al{v}$ & $\al{d}$ & $\al{a}$ & $\al{p}$ \\ 
        \hline 
        150 & 256 & 6  & 6  & 1    & 1     & 1     & 1     & 7 \\
        154 &     &    &    & 0.2  & 0.2   & 0.2   & 0.2   & 0.2 \\
        156 &     &    &    & 0.33 & 0.11  & 0.11  & 0.11  & 0.33 \\
        157 &     &    &    & 0.4  & 0.067 & 0.067 & 0.067 & 0.4 \\
        159 & 528 & 12 & 13 & 0.4  & 0.067 & 0.067 & 0.067 & 0.4 \\
        \hline
    \end{tabular}
    \caption{The model configurations of additional experiments we ran after our initial quantitative evaluation effort. We show similar hyperparemters as in table \ref{tab:quantitative}. There are additional paramter values that are not present but are used in table \ref{tab:quantitative}: \lr{} is 0.0003, \clip{} is 0.5, and \drop{} is 0.1} 
    \label{tab:qualitative-models}
    \end{center}
\end{table}

\tm{154}, which weights all expressive parameters evenly, generates performances with the global tempo slightly too fast and an extremely muddy pedal. \tm{150}, which uses the \vnet{} MSE, produces better performances with reasonable pedaling. However, the tempo is inconsistent enough that the performance loses its cohesiveness as a whole. To further test our hypothesis that the tempo and pedal are the most important expressive parameters, we increase the weights for both parameters higher than others in \tm{156}. We found that the tempo and pedal weights \tm{156} were a bit too low - specifically, the pedal is almost just as muddy as it is in \tm{150} and the tempo is still a bit too fast, albeit more consistent and cohesive than \tm{150}. We further increased the tempo and pedal weights in \tm{157} and \tm{159} and found that \tm{157} is the best sounding Transformer model\footnote{Performances of Fantaisie Impromptu best demonstrate these differences. Compare \href{https://ui.neptune.ai/richt3211/thesis/e/THESIS-154/artifacts}{\tm{154}}, \href{https://ui.neptune.ai/richt3211/thesis/e/THESIS-150/artifacts}{\tm{150}}, \href{https://ui.neptune.ai/richt3211/thesis/e/THESIS-156/artifacts}{\tm{156}}, and \href{https://ui.neptune.ai/richt3211/thesis/e/THESIS-157/artifacts}{\tm{157}} }. It is likely that were we to continue to experiment with a different configuration of the weights that we could come up with increasingly better-sounding models. 

Our last general observation is that the \vnet{} model HAN-M produces the best overall performances. In general, we feel that the tempo for HAN-M is a little too slow, but it still creates the most natural expression. This expression is most apparent in its performance of Beethoven's Piano Sonata 17% 
\footnote{See \href{https://ui.neptune.ai/richt3211/thesis/e/THESIS-162/artifacts}{\tm{162}}}, whose introduction leaves a large space for interpretation to achieve the desired listening result - rendering the score precisely as it is written results in an uninteresting performance. We found that the only model that made this performance "interesting" was the HAN-M. Although we have previously emphasized the problem using the existing quantitative metric to evaluate our models, our quantitative and qualitative evaluation of the HAN-M model indicates that it is the "best" model. The proposed Transformer architecture does not improve upon existing models. We will provide some intuition about why this is and possible model improvements for future work in chapter \ref{ch:ch8}





