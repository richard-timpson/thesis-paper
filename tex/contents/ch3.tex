\chapter{Related Work}\label{ch:ch3}
Given the understanding of both EMP and Transformers presented in section \ref{ch:ch2}, we'll now give an overview of the existing relevant research from which we will build upon. This will include a variety of different EMP models, as well as applications of the Transformer to MIR related problems. 
\section{Existing Expressive Musical Performance Generation Models}
EMP generation models fit into one of two categories, rule-based and data- based. Rule-based systems are built using a set of hardcoded rules which map were the first EMP models and provide much of the basis for other existing work. Data-driven models rely on Machine Learning (ML) and probablistic approaches 

% \begin{itemize}
%     \item KTH system \cite{friberg2006overview}. A rule-based system for expressive performance. Rules are selected through a empirical process based on human feedback. 
%     \item YQX. A Bayesian network that models timing, dynamics, and articulation \cite{widmer2009yqx}. Won the 2008 RenCon contest. 
%     \item Basis Function Models \cite{eduardo2018computational}
%     \begin{itemize}
%         \item Linear Basis Functions. Uses Least Squares regression and Bayesian models with about the same performance
%         \item Non-Linear Basis Functions. Uses standard feed-forward network. FFNN perform better than Linear models. Also uses an RNN. \rtodo{This needs more exploration. Lot of possibilities for future work}
%     \end{itemize}
%     \item Giraldo and Ramirez use several different ML algorithms, including Decision Trees, k-NN, SVM's, and FFNN to build an expressive performance generation system for improvisational Jazz guitar \cite{giraldo2016machine}. 
%     \item Moulieras and Pachet use a Maximum Entropy model to infer the underlying distribution of expressive performance and build a generation system trained from a mix of popular music. Their expressive model outperforms base models in listening tests \cite{moulieras2016maximum}. 
%     \item Jeong builds two versions of virtuosoNet, one using a recurrent hierarchical attention network (HAN) \cite{jeong2019virtuosonet}, and another using a recurrent graph network \cite{jeong2019graph}. These models are built using a dataset order of magnitudes larger than other datasets and attempt to model the expressive performance feature of the pedal, which no other model does. The code for the models is also open source so it was chosen as the starting place for this work. \rtodo[,inline]{Add more papers and expand upon the existing research a bit more. Isn't completely necessary but will be good for my overall understanding}
% \end{itemize}

\section{Datasets}
\begin{itemize}
    \item Talk about the fundamental limitations of gathering data for this problem, especially in relation to other fields \cite{friberg2006overview}. Because of this, the lack of high-quality data is limited. 
    \item The dataset used for the virtuosoNet \cite{jeong2019graph} \cite{jeong2019virtuosonet} will be the dataset used for the experiments. At the time the experiment started it was the largest publicly available dataset applicable to the EPG systems, and was chosen for use. A recent publication \cite{foscarin2020asap} builds off of the dataset used for the virtuosoNet with more sophisticated alignment and some extensions to the size (dataset is named ASAP). ASAP would be more appropriate for future use. 
    \item One of the necessary data processing tasks for EMP is the alignment between the score and performance of a given piece. Because there is always an inherent interpretation of a composition by a performer \rtodo[,inline, noinlinepar, inlinewidth=8.2cm]{Reference this in the introduction}, there is no clear mapping between any given score and performance. It remains necessary to have some sort of alignment process to match each note in the performance with its related position in the score. \rtodo[,inline]{This needs more research. Find relevant papers to cite, as well as show a diagram that makes it clear why alignment is necessary}.
\end{itemize}
