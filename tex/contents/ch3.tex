\chapter{Related Work}\label{ch:ch3}
Given the understanding of both EMP and Transformers presented in section \ref{ch:ch2}, we'll \rtodo[,inline]{Try not to use contractions. Use we will instead of we'll} now give an overview of the existing relevant research from which we will build upon. This will include a variety of different EMP models, as well as applications of the Transformer to MIR related problems. \rtodo[,inline]{Put the data section in a different place}. 

\section{Existing Expressive Musical Performance Generation Models}
EMP generation models fit into one of two categories, rule-based and data-based. Rule-based systems are built using a set of hardcoded rules which are derived using pre-existing musical knowledge and empirical studies involving human cognition. Data-driven models rely on probabilistic and machine learning methods to take an existing dataset of both scores and performances and use the performance data as a guide to learn the mapping between score features and performance features. 

\subsection{Rule Based}
The KTH system \cite{friberg2006overview} sits at the center of rule-based EMP models. Development of the KTH started in the 1980s and continued well into the 21st century. The initial idea behind the KTH system was to define a set of rules relating to the structure of a musical composition and how they affect a resulting performance, specifically with singing synthesis. The first set of rules was developed related for use in singing synthesis, and these rules were then later adapted to general musical performance. 

Since then there have been two general methods in the continued development of the KTH rule system. The first is that of \emph{analysis-by-synthesis}, which involved using the rules to synthesize musical performances that were presented to human listeners (both professional and non-professional), gathering listening feedback, and then using this feedback to modify the rules where needed. The second was an \emph{analysis-by-measurement} method. This method uses direct computation to analyze the result of a computational generated performance with an existing real performance \footnote{This falls more in line with the data-driven approaches. However, data-driven models use the performance data to directly build the model, whereas the use of real performance data in the KTH system is for evaluation purposes only. Any further updates to the model still rely on a hardcoded set of rules}. Example rules from the KTH system are found in figure \ref{fig:kth-rules} \rtodo{May need more exploration in caption}. 

\begin{figure}
    \centering
    \missingfigure{show some of the rules from the KTH paper in either a table or a figure}
    \caption{The left column shows the name of the rule, and the right column provides a language description of that rule. These are the rules that we might expect a data-based system to learn.}
    \label{fig:kth-rules}
\end{figure}

To our knowledge, the KTH rule-based system is the first sophisticated computational model for generating expressive performance, and its methods form the basis for much of the research that has been conducted since then. The explicitly defined rules in the KTH system can be thought of as the rules we might expect a data-based model to learn. \citet{widmer2002machine} shows that data-driven methods do in fact learn some of the same rules as the KTH system, but also can learn rules that are the opposite of KTH rules. As has already been discussed, the difficult nature of model evaluation may describe this phenomenon, as there is no telling which rule is more "correct" than another. Nevertheless, the KTH rule system has been an important milestone in the development of EMP models in general. 

\subsection{Data Based}\label{sec:data-based}
State of the art EMP generation models rely on existing data of actual human performance to learn the mapping between score and performance. The state of the art models are generally based either on sequential probabilistic or non-linear neural network methods\cite{cancino2018computational}, although there has been previous work with linear and non-sequential modeling. A complete overview of all relevant EMP generation models is presented in \cite{cancino2018computational} and we will not iterate them here. Instead we will describe a few models and frameworks which are relevant to our work

\subsubsection{Basis Function Models}
The first of these is a complete computational and mathematical framework for exploring EMP, and is known as the Basis Model (BM) framework\cite{eduardo2018computational}. The BM framework for EMP describes the full end-to-end process involved both the generation and analysis of musical performance, starting with a set of Basis Function Models which are used to provide score features. The BM framework also defines \emph{expressive parameters}, which are analogous to our definition of performance features as outlined in \ref{sec:performance}. Given score features which are defined by a set of basis functions as well a set of expressive parameters used to numerically define a performance, the BM framework then defines a model which can map between the score features and expressive parameters. \rtodo[,inline]{This idea needs more cohesion with the rest of the thesis. Try to provide our own mathematical definition of EMP (similarly to the way we did with neural machine translation). We could actually use the BM framework as this definition, although it may be more mathematical than we need}. 

\citet{eduardo2018computational} outlines the full mathematical definition of the BM framework, as well as the evolution of the framework and its application with specific feature and model definitions. BM models first started as simple linear non-sequential models which learned the linear relationship between a set of defined basis functions (or score feature) \rtodo{add more information about score feature} and a single expressive parameter, such as MIDI velocity. This version of the BM models each expressive parameter independently from all others, and implies that the interpretation one expressive parameter will not have an effect on the other \rtodo{Verify that footnote is correct}. \footnote{Although this is not necessarily the case in actual performance, it is a simplifying mathematical trait that makes the development and interpretation of the models simpler. All of the BM models operate under this same assumption}. Both standard least squares regression and a probabilistic Bayesian approach are used to model the linear relationship. 

As the BM framework progressed, both non-linear and sequential models were introduced in the form of deep neural networks. The non-linear model was implemented first in the form of Feed-Forward Neural Network (FFNN) was implemented first and showed an increase in goodness-of-fit as well as predictive accuracy over the standard linear models. After the FFNN came a standard RNN and was used in conjunction with the FFNN with features where time-dependent and the sequential nature of music was relevant. The recurrent non-linear model performed the best relative to all other models. 

\subsubsection{virtuosoNet}
\rtodo[,inline]{Change virtuosoNet heading to look better. Also look into creating a macro for virtuosoNet to create a typeset so that the name stands out}. 
Similarly to the BM framework, the development of virtuosoNet is gradual. The first version of the model presented in \cite{jeong2018virtuosonet} uses a recurrent hierarchical attention network (HAN) along with a novel encoder-decoder architecture specific to the EMP domain. No quantitative or qualitative evaluation results are presented at this point. The next iteration of virtuosoNet uses a similar encoder-decoder architecture but introduces an iterative sequential graph-based neural network (ISGN) that relies on the score representation as a graph data structure \cite{jeong2019graph}. The latest version presented in \cite{jeong2019virtuosonet} returns to the HAN architecture, but does so with a larger dataset as well as additional more abstract hierarchical models that are hypothesized to create better structure at the metrical level and preserve patterns across mid-level structures of the composition, in addition to learning them at the low-level.  

Both the ISGN\cite{jeong2019graph} and HAN\cite{jeong2019virtuosonet} version of virtuosoNet are trained on the same dataset (which we will describe in section \rtodo{add section}) and evaluated quantitatively using MSE and qualitatively using listening tests. In terms of quantitative evaluation, both the ISGN and HAN perform better than baseline models which remove some of the architecture complexity related to hierarchical layers. The final version of HAN reports better MSE metrics than ISGN. The qualitative evaluation with listening tests shows that both ISGN and HAN perform better than baseline models as well as better than the "deadpan" performance, which is a performance model that is statically computed using a simple set of rules and gives a somewhat robotic-sounding performance \rtodo[,inline]{Provide more explanation for the deadpan recording. May be worth it to mention in the qualitative evaluation section}. The final HAN version's qualitative evaluation includes a comparison between the HAN and the publicly available version of the BM framework model \footnote{The website for the BM model can be found \href{here}{https://basismixer.cp.jku.at/static/app.html}. At the time of this writing, the website is currently unavailable}. 

The results in \cite{jeong2019virtuosonet} show that the HAN performs better than the BM model. There are many plausible reasons that may explain the difference in results other than the HAN being a superior model to the BM, including differences in the training data for both models, bias of the qualitative method towards the HAN, and the fact that the opinion of the members of the listening test doesn't necessarily imply one model being "superior" to another. However, given the results presented by \citet{jeong2019virtuosonet}, we will assume that this version of the HAN represents the current "state of the art" in the field, if such a thing even exists. 

\rtodo[,inline]{Add section that talks about the features used for virtuosoNet}

\section{Datasets}
